\documentclass{article}

\usepackage{microtype}
\usepackage{graphicx}
\usepackage{subfigure}
\usepackage{booktabs}

\usepackage{fontawesome}
\usepackage{hyperref}
\usepackage{url}

% Attempt to make hyperref and algorithmic work together better:
\newcommand{\theHalgorithm}{\arabic{algorithm}}

\usepackage[accepted]{icml2020}

\icmltitlerunning{Spatial Transforming Network for chest X-Ray images preprocessing}

\begin{document}

\twocolumn[
\icmltitle{Spatial Transforming Network for chest Ultra-Sound images preprocessing}

\icmlsetsymbol{equal}{*}

\begin{icmlauthorlist}
\icmlauthor{Andrey Galichin}{sk}
\icmlauthor{Evgeny Gurov}{sk}
\icmlauthor{Arkadiy Vladimirov}{sk}
% \icmlauthor{Example}{equal,other}
\end{icmlauthorlist}

\icmlaffiliation{sk}{Skolkovo Institute of Science and Technology, Moscow, Russia}
% \icmlaffiliation{other}{Other affiliation}

\icmlcorrespondingauthor{Andrey Galichin}{Andrey.Galichin@skoltech.ru}

\icmlkeywords{Final Project, Machine Learning, Skoltech}

\vskip 0.3in
]
\printAffiliationsAndNotice{}  % leave blank if no need to mention equal contribution
% \printAffiliationsAndNotice{\icmlEqualContribution} % otherwise use the standard text.
\begin{abstract}

\underline{\textbf{Abstract.}} Brief and self-contained text. One paragraph (roughly 4--6 sentences) describing the motivation and key results of your project. 

Between abstract and Introduction you must insert the links to your \textbf{github repo} and \textbf{video presentation} of the project.

\end{abstract}

\underline{\textbf{Github repo:}} \href{https://github.com/bizzare-hub/Chest-Xray-alignment-using-STN.git}{https://github.com/bizzare-hub/Chest-Xray-alignment-using-STN.git}\newline

\section{Introduction}\label{introduction}

\underline{\textbf{Introduction.}} A gentle introduction to the topic of your report, deeper explanation of motivation (mentioning some recent related work on your topic). The introduction must end with the phrase \textbf{the main contributions of this report are as follows} and following concise but still very clear list of 2-4 tasks, problems, improvements, replications, experiments (listed as bullet points) that you performed in your project. This is a usual practice to make the contributions explicit to readers and reviewers. For example, see \cite{arjovsky2017wasserstein,NIPS20198433} or almost any other conference paper.

\section{Related work}\label{related_work}

\underline{\textbf{Related work}} Review of old, recent and state-of-the art methods for solving the problem students encounter in their project. At least 4-5 references should be mentioned with a brief discussion of their drawbacks and advantages.

\section{Algorithms and Models}\label{algorithms_and_models}

\underline{\textbf{Algorithms and Models}} and \underline{\textbf{Experiments and Results.}}

Two main sections describing key students' results. All the relevant content should be distributed among these sections based on the topic of project, stated goals, project plan and students' decision. In general, these sections should contain clear experimental setup and a  link to a \textbf{github repo} (again!) with a fully reproducible code. \textbf{Projects without a github repo with a reproducible code will be graded as zero.}

Students have to explicitly describe the algorithms, models, methods, approaches they used for solving their project's problem. Students should explain the motivation for choosing the models, possible benefits and drawbacks of the choice in application to their problem. The used metrics for accessing the quality of the results should also be described.

The section(s) must contain a \textbf{complete description of the datasets} used for experiments with all the required download links. This includes number of features, samples, types of features (categorial, real, pixels, etc.), description of key features, etc. If well-known datasets are used, e.g. MNIST, CIFAR, etc., it is enough to put a link to a dataset (or related paper) without a detailed description.

All the \textbf{preprocessing} and data-handling steps should be presented in these sections. Make sure to answer relevant questions, e.g. the following ones: How data was normalized? How data augmentation was done? How data was cleaned from outliers or anomalies? How the data was splitted for train, test, validation?

All \textbf{training parameters} should be listed. Which methods did you try and with which parameters (e.g. neural network architectures, weight initialization, optimizers, optimizer parameters, number of epochs, iterations, cross validation, exact number of restarts, etc.)?

\section{Experiments and Results}\label{experiments_and_results}

We highly encourage students to additionally present experimental results in a form of \textbf{tables and plots} (e.g. generated images for projects related to image generation, segmented images for project related to segmentation, table with scores for projects related to prediction, etc.). All the experimental results must be properly discussed and explained. If you experience problems with creating tables in \LaTeX, use e.g. \href{https://www.tablesgenerator.com}{this online tool} or any its analog. In Python's \textit{Pandas} library there are also methods to convert \href{https://pandas.pydata.org/pandas-docs/stable/reference/api/pandas.DataFrame.to_latex.html}{\textbf{pd.DataFrame}} to \LaTeX  .

\section{Conclusion}\label{conclusion}

\underline{\textbf{Conclusion}} Concise description of experimental results and outcomes, including possible directions for further work.

Filling appendices \ref{appendix-contrib} and  \ref{appendix-checklist} is mandatory. \textbf{Projects with empty or incomplete appendices will be graded as zero.}

\bibliography{references}
\bibliographystyle{icml2020}
\clearpage

%%%%%%%%%%%%%%%%%%%%%%%%%%%%%%%%%%%%%%%%%%%%%%%%%%%%%%%%%%%%%%%%%%%%%%%%%%%%%%%
%%%%%%%%%%%%%%%%%%%%%%%%%%%%%%%%%%%%%%%%%%%%%%%%%%%%%%%%%%%%%%%%%%%%%%%%%%%%%%%
% DELETE THIS PART. DO NOT PLACE CONTENT AFTER THE REFERENCES!
%%%%%%%%%%%%%%%%%%%%%%%%%%%%%%%%%%%%%%%%%%%%%%%%%%%%%%%%%%%%%%%%%%%%%%%%%%%%%%%
%%%%%%%%%%%%%%%%%%%%%%%%%%%%%%%%%%%%%%%%%%%%%%%%%%%%%%%%%%%%%%%%%%%%%%%%%%%%%%%

\newpage
\appendix
\section{Team member's contributions}
\label{appendix-contrib}
Explicitly stated contributions of each team member to the final project.
\subsection*{Name 1 (20\% of work)}
\begin{itemize}
    \item Reviewing literate on the topic (3 papers)
    \item Coding the main algorithm
    \item Experimenting with model parameters on MNIST dataset
    \item Preparing the GitHub Repo
    \item Preparing the Section N of this report
    \item ...
\end{itemize}

\subsection*{Name 2 (25\% of work)}
\begin{itemize}
    \item ...
\end{itemize}

\subsection*{Name 3 (55\% of work)}
\begin{itemize}
    \item ...
\end{itemize}

\clearpage
\section{Reproducibility checklist}
\label{appendix-checklist}
Answer the questions of following reproducibility checklist. If necessary, you may leave a comment.
    \begin{enumerate}
    \item A ready code was used in this project, e.g. for replication project the code from the corresponding paper was used.
    \begin{itemize}
        \item [\faCheckSquareO] Yes.
        \item [\faSquareO] No.
        \item [\faSquareO] Not applicable.
    \end{itemize}
    
    \textbf{General comment:} If the answer is \textbf{yes}, students must \underline{explicitly clarify} to which extent (e.g. which percentage of your code did you write on your own?) and which code was used.
    
    \textbf{Students' comment:} None
    \item A clear description of the mathematical setting, algorithm, and/or model is included in the report.
    \begin{itemize}
        \item [\faSquareO] Yes.
        \item [\faSquareO] No.
        \item [\faSquareO] Not applicable.
    \end{itemize}
    
    \textbf{Students' comment:} None
    
    \item A link to a downloadable source code, with specification of all dependencies, including external libraries is included in the report.
    \begin{itemize}
        \item [\faSquareO] Yes.
        \item [\faSquareO] No.
        \item [\faSquareO] Not applicable.
    \end{itemize}
    
    \textbf{Students' comment:} None
    
    \item A complete description of the data collection process, including sample size, is included in the report.
    \begin{itemize}
        \item [\faSquareO] Yes.
        \item [\faSquareO] No.
        \item [\faSquareO] Not applicable.
    \end{itemize}
    
    \textbf{Students' comment:} None
    
    \item A link to a downloadable version of the dataset or simulation environment is included in the report.
    \begin{itemize}
        \item [\faSquareO] Yes.
        \item [\faSquareO] No.
        \item [\faSquareO] Not applicable.
    \end{itemize}
    
    \textbf{Students' comment:} None
    
    \item An explanation of any data that were excluded, description of any pre-processing step are included in the report.
    \begin{itemize}
        \item [\faSquareO] Yes.
        \item [\faSquareO] No.
        \item [\faSquareO] Not applicable.
    \end{itemize}
    
    \textbf{Students' comment:} None
    
    \item An explanation of how samples were allocated for training, validation and testing is included in the report.
    \begin{itemize}
        \item [\faSquareO] Yes.
        \item [\faSquareO] No.
        \item [\faSquareO] Not applicable.
    \end{itemize}
    
    \textbf{Students' comment:} None
    
    \item The range of hyper-parameters considered, method to select the best hyper-parameter
configuration, and specification of all hyper-parameters used to generate results are included in the report.
    \begin{itemize}
        \item [\faSquareO] Yes.
        \item [\faSquareO] No.
        \item [\faSquareO] Not applicable.
    \end{itemize}
    
    \textbf{Students' comment:} None
    
    \item The exact number of evaluation runs is included.
    \begin{itemize}
        \item [\faSquareO] Yes.
        \item [\faSquareO] No.
        \item [\faSquareO] Not applicable.
    \end{itemize}
    
    \textbf{Students' comment:} None
    
    \item A description of how experiments have been conducted is included.
    \begin{itemize}
        \item [\faSquareO] Yes.
        \item [\faSquareO] No.
        \item [\faSquareO] Not applicable.
    \end{itemize}
    
    \textbf{Students' comment:} None
    
    \item A clear definition of the specific measure or statistics used to report results is included in the report.
    \begin{itemize}
        \item [\faSquareO] Yes.
        \item [\faSquareO] No.
        \item [\faSquareO] Not applicable.
    \end{itemize}
    
    \textbf{Students' comment:} None
    
    \item Clearly defined error bars are included in the report.
    \begin{itemize}
        \item [\faSquareO] Yes.
        \item [\faSquareO] No.
        \item [\faSquareO] Not applicable.
    \end{itemize}
    
    \textbf{Students' comment:} None
    
    \item A description of the computing infrastructure used is included in the report.
    \begin{itemize}
        \item [\faSquareO] Yes.
        \item [\faSquareO] No.
        \item [\faSquareO] Not applicable.
    \end{itemize}
    
    \textbf{Students' comment:} None
\end{enumerate}



\end{document}


% This document was modified from the file originally made available by
% Pat Langley and Andrea Danyluk for ICML-2K. This version was created
% by Iain Murray in 2018, and modified by Alexandre Bouchard in
% 2019 and 2020. Previous contributors include Dan Roy, Lise Getoor and Tobias
% Scheffer, which was slightly modified from the 2010 version by
% Thorsten Joachims & Johannes Fuernkranz, slightly modified from the
% 2009 version by Kiri Wagstaff and Sam Roweis's 2008 version, which is
% slightly modified from Prasad Tadepalli's 2007 version which is a
% lightly changed version of the previous year's version by Andrew
% Moore, which was in turn edited from those of Kristian Kersting and
% Codrina Lauth. Alex Smola contributed to the algorithmic style files.
